% Options for packages loaded elsewhere
\PassOptionsToPackage{unicode}{hyperref}
\PassOptionsToPackage{hyphens}{url}
%
\documentclass[
  11pt,
  ignorenonframetext,
]{beamer}
\usepackage{pgfpages}
\setbeamertemplate{caption}[numbered]
\setbeamertemplate{caption label separator}{: }
\setbeamercolor{caption name}{fg=normal text.fg}
\beamertemplatenavigationsymbolsempty
% Prevent slide breaks in the middle of a paragraph
\widowpenalties 1 10000
\raggedbottom
\setbeamertemplate{part page}{
  \centering
  \begin{beamercolorbox}[sep=16pt,center]{part title}
    \usebeamerfont{part title}\insertpart\par
  \end{beamercolorbox}
}
\setbeamertemplate{section page}{
  \centering
  \begin{beamercolorbox}[sep=12pt,center]{part title}
    \usebeamerfont{section title}\insertsection\par
  \end{beamercolorbox}
}
\setbeamertemplate{subsection page}{
  \centering
  \begin{beamercolorbox}[sep=8pt,center]{part title}
    \usebeamerfont{subsection title}\insertsubsection\par
  \end{beamercolorbox}
}
\AtBeginPart{
  \frame{\partpage}
}
\AtBeginSection{
  \ifbibliography
  \else
    \frame{\sectionpage}
  \fi
}
\AtBeginSubsection{
  \frame{\subsectionpage}
}
\usepackage{amsmath,amssymb}
\usepackage{iftex}
\ifPDFTeX
  \usepackage[T1]{fontenc}
  \usepackage[utf8]{inputenc}
  \usepackage{textcomp} % provide euro and other symbols
\else % if luatex or xetex
  \usepackage{unicode-math} % this also loads fontspec
  \defaultfontfeatures{Scale=MatchLowercase}
  \defaultfontfeatures[\rmfamily]{Ligatures=TeX,Scale=1}
\fi
\usepackage{lmodern}
\usetheme[]{metropolis}
\ifPDFTeX\else
  % xetex/luatex font selection
\fi
% Use upquote if available, for straight quotes in verbatim environments
\IfFileExists{upquote.sty}{\usepackage{upquote}}{}
\IfFileExists{microtype.sty}{% use microtype if available
  \usepackage[]{microtype}
  \UseMicrotypeSet[protrusion]{basicmath} % disable protrusion for tt fonts
}{}
\makeatletter
\@ifundefined{KOMAClassName}{% if non-KOMA class
  \IfFileExists{parskip.sty}{%
    \usepackage{parskip}
  }{% else
    \setlength{\parindent}{0pt}
    \setlength{\parskip}{6pt plus 2pt minus 1pt}}
}{% if KOMA class
  \KOMAoptions{parskip=half}}
\makeatother
\usepackage{xcolor}
\newif\ifbibliography
\setlength{\emergencystretch}{3em} % prevent overfull lines
\providecommand{\tightlist}{%
  \setlength{\itemsep}{0pt}\setlength{\parskip}{0pt}}
\setcounter{secnumdepth}{-\maxdimen} % remove section numbering
\ifLuaTeX
  \usepackage{selnolig}  % disable illegal ligatures
\fi
\IfFileExists{bookmark.sty}{\usepackage{bookmark}}{\usepackage{hyperref}}
\IfFileExists{xurl.sty}{\usepackage{xurl}}{} % add URL line breaks if available
\urlstyle{same}
\hypersetup{
  pdftitle={Encuadre},
  pdfauthor={Gerardo Martín},
  hidelinks,
  pdfcreator={LaTeX via pandoc}}

\title{Encuadre}
\subtitle{Ecología teórica}
\author{Gerardo Martín}
\date{28-07-2023}

\begin{document}
\frame{\titlepage}

\hypertarget{encuadre}{%
\section{Encuadre}\label{encuadre}}

\hypertarget{sobre-muxed}{%
\subsection{Sobre mí}\label{sobre-muxed}}

\begin{frame}{Sobre mí}
\begin{itemize}
\item
  Gerardo Martín

  \begin{itemize}
  \tightlist
  \item
    Veterinario por ULSA Bajío
  \item
    Maestro en Biología de Conservación por el INECOL A. C.
  \item
    Doctor en Salud Pública por James Cook University

    \begin{itemize}
    \tightlist
    \item
      Posdoc en Imperial College London
    \end{itemize}
  \end{itemize}
\end{itemize}
\end{frame}

\hypertarget{por-quuxe9-un-veterinario-da-mme-i}{%
\subsection{¿Por qué un veterinario da
MME-I?}\label{por-quuxe9-un-veterinario-da-mme-i}}

\begin{frame}{¿Por qué un veterinario da MME-I?}
\begin{enumerate}
\item
  Ivestigación cuantitativa

  1.1. Estadística y Matemáticas 1.2. Problemas en salud pública que
  involucran a la ecología
\item
  Las matemáticas que uso son más o menos sencillas

  2.1. Simular transmissión de enfermedades 2.2. Probar hipótesis sobre
  mecanismos de transmisión (directa vs indirecta)
\end{enumerate}

Más info en mi sitio personal (aún en construcción!):

\begin{itemize}
\tightlist
\item
  \href{https://gerardommc.gihub.io}{gerardommc.github.io}
\end{itemize}
\end{frame}

\hypertarget{sobre-ustedes}{%
\subsection{Sobre ustedes}\label{sobre-ustedes}}

\begin{frame}{Sobre ustedes}
\begin{enumerate}
\tightlist
\item
  ¿Qué esperan de la asignatura?
\item
  ¿Cómo puedo ayudar a desarrollar sus intereses particulares desde la
  Ecología Teórica?
\item
  ¿Necesitan apoyo con equipo de cómputo?
\item
  ¿Necesitan revisión de cuestiones matemáticas?
\end{enumerate}
\end{frame}

\hypertarget{sobre-la-materia}{%
\section{Sobre la materia}\label{sobre-la-materia}}

\hypertarget{cuxf3mo-se-daruxe1n-las-clases}{%
\subsection{¿Cómo se darán las
clases?}\label{cuxf3mo-se-daruxe1n-las-clases}}

\begin{frame}{¿Cómo se darán las clases?}
\begin{enumerate}
\tightlist
\item
  Todos los contenidos del curso estarán en el sitio:
\end{enumerate}

\begin{itemize}
\tightlist
\item
  \href{https://gerardommc.github.io/Ecologia-Teorica/}{gerardommc.github.io/Ecologia-Teorica/}
\end{itemize}

\begin{enumerate}
\setcounter{enumi}{2}
\item
  Clases estarán basadas en el sitio

  3.1. Google Classroom
\end{enumerate}
\end{frame}

\hypertarget{reglas}{%
\subsection{Reglas}\label{reglas}}

\begin{frame}{Reglas}
\begin{enumerate}
\item
  Hacer muchas preguntas
\item
  Decirme si paso algo por alto
\item
  Paciencia en ambas direcciones
\end{enumerate}
\end{frame}

\hypertarget{criterios-de-evaluaciuxf3n}{%
\subsection{Criterios de evaluación}\label{criterios-de-evaluaciuxf3n}}

\begin{frame}{Criterios de evaluación}
\begin{enumerate}
\tightlist
\item
  \emph{Asistencia} (25\%)
\item
  Trabajos de clase (50\%)
\item
  Examen (25\%)
\item
  Participación (2 puntos extra máximo) 4.1. En sesiones sincrónicas
  4.2. Preguntas por email en Classroom
\end{enumerate}
\end{frame}

\hypertarget{sobre-la-asistencia}{%
\subsection{\texorpdfstring{Sobre la
\emph{asistencia}}{Sobre la asistencia}}\label{sobre-la-asistencia}}

\begin{frame}{Sobre la \emph{asistencia}}
\begin{itemize}
\tightlist
\item
  No se califica presencia en salón de clases
\item
  Sí el cumplimiento de trabajos
\end{itemize}
\end{frame}

\hypertarget{sobre-el-curso}{%
\section{Sobre el curso}\label{sobre-el-curso}}

\hypertarget{quiuxe9nes-lo-impartiremos}{%
\subsection{¿Quiénes lo
impartiremos?}\label{quiuxe9nes-lo-impartiremos}}

\begin{frame}{¿Quiénes lo impartiremos?}
\begin{enumerate}
\item
  Yo (Meta, Bio, Epi)
\item
  Brenda Solórzano (Mol, Mod Gen)
\item
  Jaime Zaldívar (Plag, Int)
\item
  Carlos Yáñez (Inv)
\item
  Jorge Lopez Rocha (Pesca, caza)
\end{enumerate}
\end{frame}

\hypertarget{calificaciuxf3n-final}{%
\subsection{Calificación final}\label{calificaciuxf3n-final}}

\begin{frame}{Calificación final}
Promedio ponderado por tiempo de participación de cada profesor/a
\end{frame}

\hypertarget{contacto}{%
\subsection{Contacto}\label{contacto}}

\begin{frame}{Contacto}
\begin{itemize}
\tightlist
\item
  E-mail:
  \href{mailto:gerardo.mmc@enesmerida.unam.mx}{\nolinkurl{gerardo.mmc@enesmerida.unam.mx}}
\item
  Sitio web personal:
  \href{https://gerardommc.github.com}{gerardommc.github.io}
\end{itemize}
\end{frame}

\end{document}
