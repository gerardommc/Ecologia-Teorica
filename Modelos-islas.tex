% Options for packages loaded elsewhere
\PassOptionsToPackage{unicode}{hyperref}
\PassOptionsToPackage{hyphens}{url}
%
\documentclass[
  11pt,
  ignorenonframetext,
]{beamer}
\usepackage{pgfpages}
\setbeamertemplate{caption}[numbered]
\setbeamertemplate{caption label separator}{: }
\setbeamercolor{caption name}{fg=normal text.fg}
\beamertemplatenavigationsymbolsempty
% Prevent slide breaks in the middle of a paragraph
\widowpenalties 1 10000
\raggedbottom
\setbeamertemplate{part page}{
  \centering
  \begin{beamercolorbox}[sep=16pt,center]{part title}
    \usebeamerfont{part title}\insertpart\par
  \end{beamercolorbox}
}
\setbeamertemplate{section page}{
  \centering
  \begin{beamercolorbox}[sep=12pt,center]{part title}
    \usebeamerfont{section title}\insertsection\par
  \end{beamercolorbox}
}
\setbeamertemplate{subsection page}{
  \centering
  \begin{beamercolorbox}[sep=8pt,center]{part title}
    \usebeamerfont{subsection title}\insertsubsection\par
  \end{beamercolorbox}
}
\AtBeginPart{
  \frame{\partpage}
}
\AtBeginSection{
  \ifbibliography
  \else
    \frame{\sectionpage}
  \fi
}
\AtBeginSubsection{
  \frame{\subsectionpage}
}
\usepackage{amsmath,amssymb}
\usepackage{iftex}
\ifPDFTeX
  \usepackage[T1]{fontenc}
  \usepackage[utf8]{inputenc}
  \usepackage{textcomp} % provide euro and other symbols
\else % if luatex or xetex
  \usepackage{unicode-math} % this also loads fontspec
  \defaultfontfeatures{Scale=MatchLowercase}
  \defaultfontfeatures[\rmfamily]{Ligatures=TeX,Scale=1}
\fi
\usepackage{lmodern}
\usetheme[]{metropolis}
\ifPDFTeX\else
  % xetex/luatex font selection
\fi
% Use upquote if available, for straight quotes in verbatim environments
\IfFileExists{upquote.sty}{\usepackage{upquote}}{}
\IfFileExists{microtype.sty}{% use microtype if available
  \usepackage[]{microtype}
  \UseMicrotypeSet[protrusion]{basicmath} % disable protrusion for tt fonts
}{}
\makeatletter
\@ifundefined{KOMAClassName}{% if non-KOMA class
  \IfFileExists{parskip.sty}{%
    \usepackage{parskip}
  }{% else
    \setlength{\parindent}{0pt}
    \setlength{\parskip}{6pt plus 2pt minus 1pt}}
}{% if KOMA class
  \KOMAoptions{parskip=half}}
\makeatother
\usepackage{xcolor}
\newif\ifbibliography
\usepackage{color}
\usepackage{fancyvrb}
\newcommand{\VerbBar}{|}
\newcommand{\VERB}{\Verb[commandchars=\\\{\}]}
\DefineVerbatimEnvironment{Highlighting}{Verbatim}{commandchars=\\\{\}}
% Add ',fontsize=\small' for more characters per line
\newenvironment{Shaded}{}{}
\newcommand{\AlertTok}[1]{\textcolor[rgb]{1.00,0.00,0.00}{\textbf{#1}}}
\newcommand{\AnnotationTok}[1]{\textcolor[rgb]{0.38,0.63,0.69}{\textbf{\textit{#1}}}}
\newcommand{\AttributeTok}[1]{\textcolor[rgb]{0.49,0.56,0.16}{#1}}
\newcommand{\BaseNTok}[1]{\textcolor[rgb]{0.25,0.63,0.44}{#1}}
\newcommand{\BuiltInTok}[1]{\textcolor[rgb]{0.00,0.50,0.00}{#1}}
\newcommand{\CharTok}[1]{\textcolor[rgb]{0.25,0.44,0.63}{#1}}
\newcommand{\CommentTok}[1]{\textcolor[rgb]{0.38,0.63,0.69}{\textit{#1}}}
\newcommand{\CommentVarTok}[1]{\textcolor[rgb]{0.38,0.63,0.69}{\textbf{\textit{#1}}}}
\newcommand{\ConstantTok}[1]{\textcolor[rgb]{0.53,0.00,0.00}{#1}}
\newcommand{\ControlFlowTok}[1]{\textcolor[rgb]{0.00,0.44,0.13}{\textbf{#1}}}
\newcommand{\DataTypeTok}[1]{\textcolor[rgb]{0.56,0.13,0.00}{#1}}
\newcommand{\DecValTok}[1]{\textcolor[rgb]{0.25,0.63,0.44}{#1}}
\newcommand{\DocumentationTok}[1]{\textcolor[rgb]{0.73,0.13,0.13}{\textit{#1}}}
\newcommand{\ErrorTok}[1]{\textcolor[rgb]{1.00,0.00,0.00}{\textbf{#1}}}
\newcommand{\ExtensionTok}[1]{#1}
\newcommand{\FloatTok}[1]{\textcolor[rgb]{0.25,0.63,0.44}{#1}}
\newcommand{\FunctionTok}[1]{\textcolor[rgb]{0.02,0.16,0.49}{#1}}
\newcommand{\ImportTok}[1]{\textcolor[rgb]{0.00,0.50,0.00}{\textbf{#1}}}
\newcommand{\InformationTok}[1]{\textcolor[rgb]{0.38,0.63,0.69}{\textbf{\textit{#1}}}}
\newcommand{\KeywordTok}[1]{\textcolor[rgb]{0.00,0.44,0.13}{\textbf{#1}}}
\newcommand{\NormalTok}[1]{#1}
\newcommand{\OperatorTok}[1]{\textcolor[rgb]{0.40,0.40,0.40}{#1}}
\newcommand{\OtherTok}[1]{\textcolor[rgb]{0.00,0.44,0.13}{#1}}
\newcommand{\PreprocessorTok}[1]{\textcolor[rgb]{0.74,0.48,0.00}{#1}}
\newcommand{\RegionMarkerTok}[1]{#1}
\newcommand{\SpecialCharTok}[1]{\textcolor[rgb]{0.25,0.44,0.63}{#1}}
\newcommand{\SpecialStringTok}[1]{\textcolor[rgb]{0.73,0.40,0.53}{#1}}
\newcommand{\StringTok}[1]{\textcolor[rgb]{0.25,0.44,0.63}{#1}}
\newcommand{\VariableTok}[1]{\textcolor[rgb]{0.10,0.09,0.49}{#1}}
\newcommand{\VerbatimStringTok}[1]{\textcolor[rgb]{0.25,0.44,0.63}{#1}}
\newcommand{\WarningTok}[1]{\textcolor[rgb]{0.38,0.63,0.69}{\textbf{\textit{#1}}}}
\setlength{\emergencystretch}{3em} % prevent overfull lines
\providecommand{\tightlist}{%
  \setlength{\itemsep}{0pt}\setlength{\parskip}{0pt}}
\setcounter{secnumdepth}{-\maxdimen} % remove section numbering
\ifLuaTeX
  \usepackage{selnolig}  % disable illegal ligatures
\fi
\IfFileExists{bookmark.sty}{\usepackage{bookmark}}{\usepackage{hyperref}}
\IfFileExists{xurl.sty}{\usepackage{xurl}}{} % add URL line breaks if available
\urlstyle{same}
\hypersetup{
  pdftitle={Biogeografía de islas},
  pdfauthor={Gerardo Martín},
  hidelinks,
  pdfcreator={LaTeX via pandoc}}

\title{Biogeografía de islas}
\subtitle{Modelos}
\author{Gerardo Martín}
\date{28-07-2023}

\begin{document}
\frame{\titlepage}

\hypertarget{intro}{%
\subsection{Intro}\label{intro}}

\begin{frame}{Intro}
\begin{itemize}
\item
  Biogeografía es el estudio del efecto de la geografía en la diversidad
  biológica

  \begin{itemize}
  \item
    Analizaremos el caso especial de las islas
  \item
    Isla puede ser porción de tierra rodeada de agua, fragmento de
    bosque rodeado de cultivo, un árbol separado de otros\ldots{}
  \end{itemize}
\item
  Modelos de metapoblaciones \(\rightarrow\) parches ocupados por
  poblaciones por una sola especie
\item
  Modelos de islas \(\rightarrow\) parches ocupados por varias especies
\end{itemize}
\end{frame}

\hypertarget{esquema-del-proceso}{%
\subsection{Esquema del proceso}\label{esquema-del-proceso}}

\begin{frame}{Esquema del proceso}
\begin{center}\includegraphics[width=2.51in]{Biogeografia/Islas} \end{center}
\end{frame}

\hypertarget{alternativas-tuxe9cnicas}{%
\subsection{Alternativas técnicas}\label{alternativas-tuxe9cnicas}}

\begin{frame}{Alternativas técnicas}
\begin{enumerate}
\item
  Representar movimiento de individuos desde continente

  \begin{enumerate}
  \tightlist
  \item
    Analizar frecuencia de llegada a las islas
  \end{enumerate}
\item
  Ignorar individuos \(\rightarrow\) representar número de especies como
  población

  \begin{enumerate}
  \tightlist
  \item
    Mac Arthur y Wilson 1967
  \end{enumerate}
\end{enumerate}
\end{frame}

\hypertarget{el-modelo-de-mac-arthur-y-wilson}{%
\subsection{El modelo de Mac Arthur y
Wilson}\label{el-modelo-de-mac-arthur-y-wilson}}

\begin{frame}{El modelo de Mac Arthur y Wilson}
\begin{enumerate}
\item
  Número de especies \(\rightarrow\) balance entre colonización y
  extinción
\item
  \(\forall\) spp tienen misma prob de llegar a isla
\item
  Sólo cuentan las colonizaciones, llegada de spp nuevas
\item
  Probabilidad de extinción es constante
\item
  Probabilidad de extinción de cualquier especie aumenta con el número
  de especies en isla
\end{enumerate}
\end{frame}

\hypertarget{el-modelo-de-mac-arthur-y-wilson-1967}{%
\subsection{El modelo de Mac Arthur y Wilson
(1967)}\label{el-modelo-de-mac-arthur-y-wilson-1967}}

\begin{frame}{El modelo de Mac Arthur y Wilson (1967)}
\begin{figure}

{\centering \includegraphics[width=4.73in]{Biogeografia/MacArthur} 

}

\caption{Relaciones cuantitativas entre inmigración, extinción, especies presentes y posibles.}\label{fig:unnamed-chunk-2}
\end{figure}
\end{frame}

\hypertarget{donde}{%
\subsection{Donde \ldots{}}\label{donde}}

\begin{frame}{Donde \ldots{}}
\begin{align}
I &= I_x - (I_x/P)R \\
E &= (E_x/P)R
\end{align}

\begin{itemize}
\item
  \(I =\) inmigración
\item
  \(E =\) extinción
\item
  \(P =\) número de especies que pueden colonizar la isla
\item
  \(R =\) número de especies que habitan la isla
\end{itemize}
\end{frame}

\hypertarget{section}{%
\subsection{\ldots{}}\label{section}}

\begin{frame}{\ldots{}}
\begin{itemize}
\item
  \(I_x\) es la tasa máxima de colonización
\item
  \(E_x\) es la tasa máxima de extinción
\end{itemize}

Por lo tanto el modelo completo es:

\[R_{t+1} = R_t + I_x - (I_x/P)R_t - (E_x/P)R_t\] \#\# Características
del modelo de Mac Arthur y Wilson

Tiempo discreto
\end{frame}

\hypertarget{los-muxe9todos}{%
\subsection{Los métodos}\label{los-muxe9todos}}

\begin{frame}{Los métodos}
\begin{itemize}
\item
  Crearemos un conjunto de parches de hábitat
\item
  Parches estarán conectados con parches adyacentes
\item
  Conjunto de parches: matriz de coordenadas geográficas
\item
  Conexiones entre parches: matriz que indica pares de parches
  conectados
\end{itemize}
\end{frame}

\hypertarget{creando-la-matriz-de-coordenadas}{%
\subsection{Creando la matriz de
coordenadas}\label{creando-la-matriz-de-coordenadas}}

\begin{frame}[fragile]{Creando la matriz de coordenadas}
\begin{Shaded}
\begin{Highlighting}[]
\NormalTok{parches }\OtherTok{\textless{}{-}} \FunctionTok{expand.grid}\NormalTok{(}\AttributeTok{x =} \DecValTok{1}\SpecialCharTok{:}\DecValTok{4}\NormalTok{, }\AttributeTok{y =} \DecValTok{1}\SpecialCharTok{:}\DecValTok{4}\NormalTok{)}
\FunctionTok{plot}\NormalTok{(parches}\SpecialCharTok{$}\NormalTok{x, parches}\SpecialCharTok{$}\NormalTok{y)}
\end{Highlighting}
\end{Shaded}

\begin{center}\includegraphics{Modelos-islas_files/figure-beamer/unnamed-chunk-3-1} \end{center}
\end{frame}

\hypertarget{matriz-de-adyacencias}{%
\subsection{Matriz de adyacencias}\label{matriz-de-adyacencias}}

\begin{frame}{Matriz de adyacencias}
Ponemos esto y lo otro\ldots{}
\end{frame}

\end{document}
