% Options for packages loaded elsewhere
\PassOptionsToPackage{unicode}{hyperref}
\PassOptionsToPackage{hyphens}{url}
%
\documentclass[
  11pt,
  ignorenonframetext,
]{beamer}
\usepackage{pgfpages}
\setbeamertemplate{caption}[numbered]
\setbeamertemplate{caption label separator}{: }
\setbeamercolor{caption name}{fg=normal text.fg}
\beamertemplatenavigationsymbolsempty
% Prevent slide breaks in the middle of a paragraph
\widowpenalties 1 10000
\raggedbottom
\setbeamertemplate{part page}{
  \centering
  \begin{beamercolorbox}[sep=16pt,center]{part title}
    \usebeamerfont{part title}\insertpart\par
  \end{beamercolorbox}
}
\setbeamertemplate{section page}{
  \centering
  \begin{beamercolorbox}[sep=12pt,center]{part title}
    \usebeamerfont{section title}\insertsection\par
  \end{beamercolorbox}
}
\setbeamertemplate{subsection page}{
  \centering
  \begin{beamercolorbox}[sep=8pt,center]{part title}
    \usebeamerfont{subsection title}\insertsubsection\par
  \end{beamercolorbox}
}
\AtBeginPart{
  \frame{\partpage}
}
\AtBeginSection{
  \ifbibliography
  \else
    \frame{\sectionpage}
  \fi
}
\AtBeginSubsection{
  \frame{\subsectionpage}
}
\usepackage{amsmath,amssymb}
\usepackage{iftex}
\ifPDFTeX
  \usepackage[T1]{fontenc}
  \usepackage[utf8]{inputenc}
  \usepackage{textcomp} % provide euro and other symbols
\else % if luatex or xetex
  \usepackage{unicode-math} % this also loads fontspec
  \defaultfontfeatures{Scale=MatchLowercase}
  \defaultfontfeatures[\rmfamily]{Ligatures=TeX,Scale=1}
\fi
\usepackage{lmodern}
\usetheme[]{metropolis}
\ifPDFTeX\else
  % xetex/luatex font selection
\fi
% Use upquote if available, for straight quotes in verbatim environments
\IfFileExists{upquote.sty}{\usepackage{upquote}}{}
\IfFileExists{microtype.sty}{% use microtype if available
  \usepackage[]{microtype}
  \UseMicrotypeSet[protrusion]{basicmath} % disable protrusion for tt fonts
}{}
\makeatletter
\@ifundefined{KOMAClassName}{% if non-KOMA class
  \IfFileExists{parskip.sty}{%
    \usepackage{parskip}
  }{% else
    \setlength{\parindent}{0pt}
    \setlength{\parskip}{6pt plus 2pt minus 1pt}}
}{% if KOMA class
  \KOMAoptions{parskip=half}}
\makeatother
\usepackage{xcolor}
\newif\ifbibliography
\usepackage{longtable,booktabs,array}
\usepackage{calc} % for calculating minipage widths
\usepackage{caption}
% Make caption package work with longtable
\makeatletter
\def\fnum@table{\tablename~\thetable}
\makeatother
\usepackage{graphicx}
\makeatletter
\def\maxwidth{\ifdim\Gin@nat@width>\linewidth\linewidth\else\Gin@nat@width\fi}
\def\maxheight{\ifdim\Gin@nat@height>\textheight\textheight\else\Gin@nat@height\fi}
\makeatother
% Scale images if necessary, so that they will not overflow the page
% margins by default, and it is still possible to overwrite the defaults
% using explicit options in \includegraphics[width, height, ...]{}
\setkeys{Gin}{width=\maxwidth,height=\maxheight,keepaspectratio}
% Set default figure placement to htbp
\makeatletter
\def\fps@figure{htbp}
\makeatother
\setlength{\emergencystretch}{3em} % prevent overfull lines
\providecommand{\tightlist}{%
  \setlength{\itemsep}{0pt}\setlength{\parskip}{0pt}}
\setcounter{secnumdepth}{-\maxdimen} % remove section numbering
\ifLuaTeX
  \usepackage{selnolig}  % disable illegal ligatures
\fi
\IfFileExists{bookmark.sty}{\usepackage{bookmark}}{\usepackage{hyperref}}
\IfFileExists{xurl.sty}{\usepackage{xurl}}{} % add URL line breaks if available
\urlstyle{same}
\hypersetup{
  pdftitle={Biogeografía de islas},
  pdfauthor={Gerardo Martín},
  hidelinks,
  pdfcreator={LaTeX via pandoc}}

\title{Biogeografía de islas}
\subtitle{Función de incidencia de Hanski}
\author{Gerardo Martín}
\date{28-07-2023}

\begin{document}
\frame{\titlepage}

\hypertarget{intro}{%
\subsection{Intro}\label{intro}}

\begin{frame}{Intro}
Modelos anteriores representan:

\begin{itemize}
\item
  Número de especies como función de:

  \begin{itemize}
  \item
    Especies continentales
  \item
    Riesgo de extinción
  \item
    Probailidad de inmigración
  \end{itemize}
\item
  Determinantes geográficos del número de especies

  \begin{itemize}
  \tightlist
  \item
    Áreas y Distancias
  \end{itemize}
\end{itemize}
\end{frame}

\hypertarget{un-marco-para-anuxe1lisis-de-datos}{%
\subsection{Un marco para análisis de
datos}\label{un-marco-para-anuxe1lisis-de-datos}}

\begin{frame}{Un marco para análisis de datos}
\begin{itemize}
\item
  Levins y MacArthur y Wilson ignoran características de islas
\item
  No permiten estimar efectos sobre número de especies
\end{itemize}

MacArthur y Wilson (1963):
\(\uparrow \mathrm{Area} \rightarrow \mathrm{Extinción} \downarrow\)

Hanski propuso modelo para relacionarlos
\end{frame}

\hypertarget{el-modelo-de-incidencia-de-hanski-1994}{%
\subsection{El modelo de incidencia de Hanski
(1994)}\label{el-modelo-de-incidencia-de-hanski-1994}}

\begin{frame}{El modelo de incidencia de Hanski (1994)}
\begin{itemize}
\item
  Ocupación es función de colonización y extinción
\item
  Modelo representa probabilidad de transición:
\end{itemize}

\[\mathrm{Vacío} \rightarrow \mathrm{Ocupado}\]

\begin{itemize}
\tightlist
\item
  De modo que:
\end{itemize}

\begin{align}
\mathrm{Estado}_t &= \mathrm{Vacío} \\
\mathrm{Estado}_{t+1} &= \mathrm{Ocupado}
\end{align}
\end{frame}

\hypertarget{esquema-del-fenuxf3meno}{%
\subsection{Esquema del fenómeno}\label{esquema-del-fenuxf3meno}}

\begin{frame}{Esquema del fenómeno}
\begin{center}\includegraphics{Hanski_files/figure-beamer/unnamed-chunk-1-1} \end{center}
\end{frame}

\hypertarget{paruxe1metros}{%
\subsection{Parámetros}\label{paruxe1metros}}

\begin{frame}{Parámetros}
\begin{itemize}
\item
  \(C_i\) es la probabilidad de ser colonizado en período \(t\)
\item
  \(E_i\) es la probabilidad de sufrir una extinción
\item
  \(1-C_i\) es pa probabilidad de permanecer ocupado
\item
  \(1 - E_i\) es la probabilidad de permanecer vacío
\end{itemize}
\end{frame}

\hypertarget{probabilidad-de-que-parche-estuxe9-ocupado}{%
\subsection{Probabilidad de que parche esté
ocupado}\label{probabilidad-de-que-parche-estuxe9-ocupado}}

\begin{frame}{Probabilidad de que parche esté ocupado}
\begin{equation}
J_i = \frac{C_i}{C_i + E_i}
\end{equation}

Si \(C_i = 0.3\) y \(E_i = 0.5\)

\begin{equation}
J_i = \frac{0.3}{0.3 + 0.5} = 0.375
\end{equation}
\end{frame}

\hypertarget{derivaciuxf3n-de-j_i}{%
\subsection{\texorpdfstring{Derivación de
\(J_i\)}{Derivación de J\_i}}\label{derivaciuxf3n-de-j_i}}

\begin{frame}{Derivación de \(J_i\)}
Se parte del modelo de lluvia de propágulos (única fuente de especies es
el continente):

\[\frac{dp}{dt} = c(1-p) - ep\] Donde las condiciones de equilibio son:

\[p^* = \frac{c}{c+e}\]
\end{frame}

\hypertarget{matriz-de-transiciones}{%
\subsection{Matriz de transiciones}\label{matriz-de-transiciones}}

\begin{frame}[fragile]{Matriz de transiciones}
\begin{longtable}[]{@{}lll@{}}
\caption{Primera fila es la probabilidad asociada a \emph{t}. Segunda
fila a \emph{t+1}.}\tabularnewline
\toprule\noalign{}
& Vacío & Ocupado \\
\midrule\noalign{}
\endfirsthead
\toprule\noalign{}
& Vacío & Ocupado \\
\midrule\noalign{}
\endhead
Vacío & 0.7 & 0.5 \\
Ocupado & 0.3 & 0.5 \\
\bottomrule\noalign{}
\end{longtable}

\(J_i\) es la probabilidad a largo plazo de ocupación, por lo tanto el
punto de equilibrio. La sitribución estable de los valores propios
\(\lambda\) es:

\begin{verbatim}
## [1] 0.625 0.375
\end{verbatim}
\end{frame}

\hypertarget{estimaciuxf3n-de-la-probabilidad-de-extincion-e_i}{%
\subsection{\texorpdfstring{Estimación de la probabilidad de extincion
(\(E_i\))}{Estimación de la probabilidad de extincion (E\_i)}}\label{estimaciuxf3n-de-la-probabilidad-de-extincion-e_i}}

\begin{frame}{Estimación de la probabilidad de extincion (\(E_i\))}
\begin{itemize}
\item
  Se determina como función del Área (\(A_i\))

  \begin{itemize}
  \tightlist
  \item
    En áreas grandes \(E_i\) es pequeño \begin{equation}
    E_i = \left\{ \begin{aligned}
    \frac{e}{A^x} \mathrm{\ si\ } A_i > e^{1/x}\\
    1 \mathrm{\ si\ } A_i \leq e^{1/x}
    \end{aligned} \right.
    \end{equation}
  \end{itemize}
\end{itemize}

Donde \(e\) es un parámetro a estimar (no es la cte de Euler).
\end{frame}

\hypertarget{ejemplo-del-efecto-del-uxe1rea-sobre-e_i}{%
\subsection{\texorpdfstring{Ejemplo del efecto del área sobre
\(E_i\)}{Ejemplo del efecto del área sobre E\_i}}\label{ejemplo-del-efecto-del-uxe1rea-sobre-e_i}}

\begin{frame}{Ejemplo del efecto del área sobre \(E_i\)}
\includegraphics{Hanski_files/figure-beamer/unnamed-chunk-4-1.pdf}
\end{frame}

\hypertarget{probabilidad-de-colonizaciuxf3n-c_i}{%
\subsection{\texorpdfstring{Probabilidad de colonización
(\(C_i\))}{Probabilidad de colonización (C\_i)}}\label{probabilidad-de-colonizaciuxf3n-c_i}}

\begin{frame}{Probabilidad de colonización (\(C_i\))}
\begin{itemize}
\tightlist
\item
  Es función de migrantes y distancia de tierra continental u otros
  parches:
\end{itemize}

\[C_i = \frac{1}{1 + \left(\frac{y'}{S_i}\right)^2}\]

\(y'\) es la habilidad colonizadora de las especies

\(S_i\) es una de aislamiento del parche \(i\)
\end{frame}

\hypertarget{fuxf3rmula-para-s_i}{%
\subsection{\texorpdfstring{Fórmula para
\(S_i\)}{Fórmula para S\_i}}\label{fuxf3rmula-para-s_i}}

\begin{frame}{Fórmula para \(S_i\)}
\[S_i = \sum_{j = i}^{n} p_j \exp(- \alpha d_{ij}) A_j\]

\(n\) número total de parches \(j\) que son hábitats de las especies
migrantes

\(p_j\) es el estado de ocupación de cada parche

\(d_{ij}\) es la distancia lineal entre parche \(i\) y el \(j\)

\(\alpha\) es el efecto de la distancia entre \(i\) y \(j\)

\(A_j\) es el área de \(j\) \(\therefore\) índice de tamaño poblacional
\end{frame}

\hypertarget{ejemplo-del-efecto-de-d_ij-y-a_j}{%
\subsection{\texorpdfstring{Ejemplo del efecto de \(d_{ij}\) y
\(A_j\)}{Ejemplo del efecto de d\_\{ij\} y A\_j}}\label{ejemplo-del-efecto-de-d_ij-y-a_j}}

\begin{frame}{Ejemplo del efecto de \(d_{ij}\) y \(A_j\)}
\begin{center}\includegraphics{Hanski_files/figure-beamer/unnamed-chunk-5-1} \end{center}
\end{frame}

\hypertarget{combinando-c_i-y-e_i-para-obtener-j_i}{%
\subsection{\texorpdfstring{Combinando \(C_i\) y \(E_i\) para obtener
\(J_i\)}{Combinando C\_i y E\_i para obtener J\_i}}\label{combinando-c_i-y-e_i-para-obtener-j_i}}

\begin{frame}{Combinando \(C_i\) y \(E_i\) para obtener \(J_i\)}
\begin{equation}
J_i = \frac{1}{1 + \left( 1 + \left[ \frac{y'}{S_i} \right]^2 \right) \frac{e}{A_i^x}}
\end{equation}

Para lo cual necesitamos los siguientes datos:

\begin{itemize}
\item
  \(A_i\), las áreas de cada parche
\item
  ubicación geográfica de cada parche que recibe (\(i\)) ó emite (\(j\))
  especies
\item
  presencia ó ausencia en cada parche
\item
  Parámetro de distancia \(\alpha\) (se estima con regresión no lineal)
\end{itemize}
\end{frame}

\hypertarget{referencias}{%
\subsection{Referencias}\label{referencias}}

\begin{frame}{Referencias}
\begin{itemize}
\item
  Ilkka Hanski (1994).
  \href{https://doi.org/10.1016/0169-5347(94)90177-5}{Patch-occupancy
  dynamics in fragmented landscapes}. \emph{Trends in Ecology and
  Evolution}.
\item
  Robert MacArthur et al.~(1963).
  \href{https://doi.org/10.2307/2407089}{An Equilibrium Theory of
  Insular Zoogeography}. \emph{Evolution}.
\item
  Nicolas Gotelli y B J McGill (2006).
  \href{https://doi.org/10.1111/j.2006.0906-7590.04714.x}{Null Versus
  Neutral Models: What's the Difference?}. \emph{Ecography}.
\item
  Hank Stevens (2023).
  \href{https://hankstevens.github.io/Primer-of-Ecology/index.html}{Primer
  of Ecology using R}.
\end{itemize}
\end{frame}

\end{document}
